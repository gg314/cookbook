\section{Carrot-Tahini Soup with Coriander, Turmeric, and Lemon}
\begin{recipe}
	\yield{4 to 6 servings}
	\cooktime{1 hour, 10 minutes}
	\source{NYTimes Cooking}
	% https://cooking.nytimes.com/recipes/1013106-carrot-tahini-soup-with-coriander-turmeric-and-lemon
	
	\pre{This soup has a flavorful ``kick'' to it. Keeps very well in the freezer.}
	
	\ingredients{
		2				& \T olive oil, more for pita chips \\
		1				& fat leek (or 2 slim), white part only, thinly sliced \\
		6				& garlic cloves, minced \\
		1				& kosher salt, more to taste \\
		\sfrac{1}{2}	& \t ground black pepper \\
		\sfrac{1}{2}	& \t ground coriander \\
		\sfrac{1}{4}	& \t turmeric powder \\
		1	& pinch cayenne \\
		1 \sfrac{1}{2}	& \lbs carrots, peeled and sliced \sfrac{1}{2}" thick \\
		1				& \qt vegetable broth \\
		2				& sprigs fresh thyme (optional) \\
		2				& pita breads, cut into 16 wedges \\
		\sfrac{1}{3}	& \cup tahini \\
						& lemon juice, fresh, to taste \\
						& lemon zest, grated for garnish \\
		\sfrac{1}{3}	& \cup fresh cilantro or mint, chopped
	}
	

In a soup pot over medium heat, pour in oil, then add leek and sauté until translucent, about 4 minutes. Add garlic, salt, pepper, coriander, turmeric and cayenne, and cook until garlic is fragrant, about 1 minute.

Add carrots and stir to coat them with leek mixture. Cook, stirring, for 3 minutes, then add broth, thyme, if using, and 2 cups water. Bring to a simmer, partly cover and cook until carrots are very tender, about 25 minutes.

Meanwhile, heat oven to 400 degrees. Brush pita wedges with a little olive oil and sprinkle with salt. Spread them on a baking sheet and bake until brown and crisp, about 10 to 12 minutes. Let cool.
When carrots are tender, turn off heat and let soup cool for 10 minutes. Remove thyme branches and stir in tahini. Using either an immersion blender, standard blender or food processor, purée soup until smooth. 

Return to pot and reheat if necessary. Taste and add lemon juice and more salt if desired.

Serve soup garnished with the lemon zest, cilantro and pita wedges.


	
\end{recipe}
